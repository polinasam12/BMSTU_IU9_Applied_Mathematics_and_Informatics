\documentclass [12pt]{article}


\usepackage{ucs}
\usepackage[utf8x]{inputenc} %Поддержка UTF8
\usepackage{cmap} % Улучшенный поиск русских слов в полученном pdf-файле
\usepackage[english,russian]{babel} %Пакет для поддержки русского и английского языка
\usepackage{graphicx} %Поддержка графиков
\usepackage{float} %Поддержка float-графиков
\usepackage[left=20mm,right=15mm, top=20mm,bottom=20mm,bindingoffset=0cm]{geometry}
\usepackage{mathtools}
\usepackage{setspace,amsmath}
\usepackage{amsmath,amssymb}
\usepackage{dsfont}
\usepackage{tabularx}
\DeclarePairedDelimiter{\abs}{\lvert}{\rvert}
\renewcommand{\baselinestretch}{1.2}

\usepackage{color}
\definecolor{deepblue}{rgb}{0,0,0.5}
\definecolor{deepred}{rgb}{0.6,0,0}
\definecolor{deepgreen}{rgb}{0,0.5,0}
\definecolor{gray}{rgb}{0.5,0.5,0.5}

\DeclareFixedFont{\ttb}{T1}{txtt}{bx}{n}{12} % for bold
\DeclareFixedFont{\ttm}{T1}{txtt}{m}{n}{12}  % for normal

\usepackage{listings}


\lstset{
	language=Python,
	basicstyle=\ttm,
	otherkeywords={self},             % Add keywords here
	keywordstyle=\ttb\color{deepblue},
	emph={MyClass,__init__},          % Custom highlighting
	emphstyle=\ttb\color{deepred},    % Custom highlighting style
	stringstyle=\color{deepgreen},
	frame=tb,                         % Any extra options here
	showstringspaces=false            %
}

\usepackage{hyperref}
\usepackage{wasysym}

\hypersetup{
    bookmarks=true,         % show bookmarks bar?
    unicode=false,          % non-Latin characters in Acrobat’s bookmarks
    pdftoolbar=true,        % show Acrobat’s toolbar?
    pdfmenubar=true,        % show Acrobat’s menu?
    pdffitwindow=false,     % window fit to page when opened
    pdfstartview={FitH},    % fits the width of the page to the window
    pdftitle={My title},    % title
    pdfauthor={Author},     % author
    pdfsubject={Subject},   % subject of the document
    pdfcreator={Creator},   % creator of the document
    pdfproducer={Producer}, % producer of the document
    pdfkeywords={keyword1} {key2} {key3}, % list of keywords
    pdfnewwindow=true,      % links in new PDF window
    colorlinks=true,       % false: boxed links; true: colored links
    linkcolor=black,          % color of internal links (change box color with linkbordercolor)
    citecolor=green,        % color of links to bibliography
    filecolor=magenta,      % color of file links
    urlcolor=cyan           % color of external links
}


\title{}
\date{}
\author{}

\begin{document}
\begin{titlepage}
\thispagestyle{empty}
\begin{center}
Федеральное государственное бюджетное образовательное учреждение высшего профессионального образования \\Московский государственный технический университет имени Н.Э. Баумана

\end{center}
\bigskip
\begin{flushleft}
Факультет: \underline{Информатика и системы управления}\\
Кафедра: \underline{Теоретическая информатика и компьютерные технологии}
\end{flushleft}
\vfill
\centerline{\large{Лабораторная работа №5}}
\centerline{\large{по курсу <<Численные методы>>}}
\centerline{\large{<<Метод наименьших квадратов. Аппроксимация алгебраическими многочленами>>}}
\vfill
\hfill\parbox{5cm} {
           Выполнила:\\
           студентка группы ИУ9-62Б \hfill \\
           Самохвалова П. С.\hfill \medskip\\
           Проверила:\\
           Домрачева А. Б.\hfill
       }
\centerline{Москва, 2023}
\clearpage
\end{titlepage}

\textsc{\textbf{Цель:}}

Анализ метода наименьших квадратов для решения задачи аппроксимации алгебраическими многочленами.

\textsc{\textbf{Постановка задачи:}}

\textbf{Дано:}  Функция $y_i = f(x_i),  i = \overline{0,n}$, значения $y_i$ могут иметь случайные ошибки.

\begin{table}[h]
\begin{center}
\begin{tabular}{|c|c|c|c|c|}
\hline
$x_1$ & $x_2$ & ... & $x_{n-1}$ & $x_n$ \\
\hline
$y_1$ & $y_2$ & ... & $y_{n-1}$ & $y_n$ \\
\hline
\end{tabular}
\end{center}
\end{table}

\textbf{Найти:} Гладкую аналитическую функцию  $z(x)$, доставляющую наименьшее значение величине $$ SKU=\sqrt{\sum\limits_{i = 0}^n{(z(x_i) - y_i)^2}}$$ Эту величину называют среднеквадратичным уклонением функции $z(x)$ от системы узлов. Описанный подход к решению задачи приближения функции - методом наименьших квадратов.

\textbf{Тестовый пример:}

Вариант 21

$$f(x) = y$$

\begin{table}[h]
\begin{center}
\begin{tabular}{|c|c|c|c|c|c|c|c|c|c|c|}
\hline
x & 1 & 1.5 & 2 & 2.5 & 3 & 3.5 & 4 & 4.5 & 5\\
\hline
y & 0.86 & 0.97 & 0.65 & 0.75 & 1.60 & 0.65 & 1.34 & 1.62 & 1.01\\
\hline
\end{tabular}
\end{center}
\end{table}

\textbf{Задание:}

\begin{enumerate}
    \item Аппроксимировать данную функцию по методу наименьших квадратов многочленом третьей степени (m = 4). Найти:\\
    а) Матрицу A и столбец b;\\
    б) Набор коэффициентов $\lambda_1$, $\lambda_2$, $\lambda_3$, $\lambda_4$;\\
    в) Среднеквадратичное отклонение $\Delta$ и относительную ошибку $\delta$;\\
    г) Значения аппроксимирующего многочлена z(x) в средних точках отрезков между узловыми точками.
    \item То же для решения задачи Коши, полученного в предыдущей лабораторной работе.
\end{enumerate}

\textbf{Описание метода:}

Как правило, $z(x)$ отыскивают в виде линейной комбинации заданных функций $$z(x) = \lambda_{1}\varphi_{1}(x) + ... + \lambda_m\varphi_m(x)$$
Параметры $\lambda_i, i = \overline{1,m}$ являются решениями линейной системы наименьших квадратов $$ A\lambda = b, $$
где $\lambda$ - столбец параметров $\lambda_i$ . $A = (a_{ij})$ - симметричная положительно определенная матрица (матрица Грама) с коэффициентами $ a_{ij} = \sum\limits_{k = 0}^n{\varphi_i(x_i)\varphi_j(x_k)}$; b - столбец правой части системы, $ b_i = \sum\limits_{k = 0}^n{\varphi_i(x_k)y_k}, \quad i,j = \overline{1,m} $


Если приближаемая функция достаточно глакая, хотя вид ее и неизвестен, аппроксимирующую функцию нередко ищут в виде алгебраического многочлена $$ z(x) = \lambda_{1} + \lambda_{2}x + ... + \lambda_{m}x^{m-1}.$$
Тогда $\varphi_i = x^{i-1}$ и элементы матрица Грама получают по формулам $$ a_{ij} = \sum\limits_{k = 0}^n{x_{k}^{i + j - 2}},$$
а свободные члены - $$b_i = \sum\limits_{k = 0}^n{y_{k}^{i - 1}}, \quad i, j = \overline{1,m} $$

Абсолютной погрешностью аппроксимации служит среднеквадратичное отклонение (СКО): $$ \Delta = \frac{SKU}{\sqrt{n}} = \frac{1}{\sqrt{n}}\sqrt{\sum\limits_{k = 0}^n{(y_k - \lambda_{1} - \lambda_{2}x_k - ... -\lambda_{m}x_{k}^{m-1})^2}},$$
относительная ошибка $$ \delta = \frac{\Delta}{||y||} = \frac{\Delta}{\sqrt{\sum\limits_{k = 0}^n{y_{k}^2}}} $$

Листинг 1. Метод наименьших квадратов. Аппроксимация алгебраическими многочленами

\begin{lstlisting}[language=python]
def mnk(n, x, y, m):
    a = []
    for i in range(m):
        a.append([0] * m)
    b = [0] * m

    for i in range(m):
        for j in range(m):
            for k in range(n + 1):
                a[i][j] += x[k] ** (i + j)

    for i in range(m):
        for k in range(n + 1):
            b[i] += y[k] * (x[k] ** i)

    print("A =")

    for i in range(m):
        for j in range(m):
            print("{:20} ".format(str(a[i][j])), end="")
        print()

    print()

    print("b =")

    for i in range(m):
        print(b[i])

    print()

    t = []
    for i in range(m):
        t.append([0] * m)

    t_tr = []

    for i in range(m):
        t_tr.append([0] * m)

    t[0][0] = a[0][0] ** 0.5

    for j in range(1, m):
        t[0][j] = a[0][j] / t[0][0]

    for i in range(1, m):
        for j in range(m):
            if i == j:
                s = 0
                for k in range(i):
                    s += t[k][i] ** 2
                t[i][i] = (a[i][i] - s) ** 0.5
            elif i < j:
                s = 0
                for k in range(i):
                    s += t[k][i] * t[k][j]
                t[i][j] = (a[i][j] - s) / t[i][i]

    for i in range(m):
        for j in range(m):
            t_tr[i][j] = t[j][i]

    xr = [0] * m
    yr = [0] * m

    yr[0] = b[0] / t[0][0]

    for i in range(1, m):
        s = 0
        for k in range(i):
            s += t[k][i] * yr[k]
        yr[i] = (b[i] - s) / t[i][i]

    xr[m - 1] = yr[m - 1] / t[m - 1][m - 1]

    for i in range(m - 2, -1, -1):
        s = 0
        for k in range(i + 1, m):
            s += t[i][k] * xr[k]
        xr[i] = (yr[i] - s) / t[i][i]

    la = xr[:]

    print("lambda =")

    for i in range(m):
        print(la[i])

    print()

    s = 0

    for k in range(n + 1):
        s1 = 0
        for q in range(m):
            s1 += la[q] * (x[k] ** q)
        s += (y[k] - s1) ** 2

    abs_delta = (1 / (n ** 0.5)) * (s ** 0.5)

    s = 0

    for k in range(n + 1):
        s += y[k] ** 2

    otn_delta = abs_delta / (s ** 0.5)

    print("abs_delta =")
    print(abs_delta)

    print()

    print("otn_delta = ")
    print(otn_delta)

    print()

    ym = [0] * n

    for i in range(n):
        xm = (x[i] + x[i + 1]) / 2
        s = 0
        for k in range(m):
            s += (la[k] * (xm ** k))
        ym[i] = s

    print("ym = ")
    for i in range(n):
        print(ym[i])


n = 8
x = [1, 1.5, 2, 2.5, 3, 3.5, 4, 4.5, 5]
y = [0.86, 0.97, 0.65, 0.75, 1.60, 0.65, 1.34, 1.62, 1.01]
m = 4

print("1.")
print()
mnk(n, x, y, m)
print()

n = 10
x = [0.0, 0.1, 0.2, 0.30000000000000004, 0.4, 0.5, 0.6000000000000001,
     0.7000000000000001, 0.8, 0.9, 1.0]
y = [1.0, 1.205004334722476, 1.420072088955231, 1.6453790142373428,
     1.881243039949921, 2.128146809304331, 2.3867612737855444,
     2.6579703947081676, 2.9428970193919906, 3.242930020784433,
     3.5597528132669414]
m = 4

print("2.")
print()
mnk(n, x, y, m)

\end{lstlisting}


\textsc{\textbf{Результаты работы:}}

В результате работы программы на тестовом примере получаем:

\[
  A =
  \begin{pmatrix}
    9.0 & 27.0 & 96.0 & 378.0\\
    27.0 & 96.0 & 378.0 & 1583.25\\
    96.0 & 378.0 & 1583.25 & 6900.75\\
    378.0 & 1583.25 & 6900.75 & 30912.5625
  \end{pmatrix}
\]
\[
  b =
  \begin{pmatrix}
    9.450000000000001 \\
    30.265000000000004 \\
    112.1875 \\
    451.75375
  \end{pmatrix}
\]
\[
  \lambda =
  \begin{pmatrix}
    1.913730158729999 \\
    -1.544129389129182 \\
    0.6322510822510062 \\
    -0.07084175084174252
  \end{pmatrix}
\]

СКО $\Delta = 0.3179591701941208$

Относительная ошибка $\delta = 0.09548667876045708$

Таблица значений аппроксимирующего многочлена z(x) в средних точках отрезков между узловыми точками

\begin{center}
\begin{tabular}{|c|c|}
    \hline
    $x_n$ & $z(x_n)$ \\ \hline
    1.25 & 0.8330979437229404 \\ \hline
    1.75 & 0.7681051587301733 \\ \hline
    2.25 & 0.8332783189033353 \\ \hline
    2.75 & 0.9754861111111188 \\ \hline
    3.25 & 1.1415972222222188 \\ \hline
    3.75 & 1.278480339105326 \\ \hline
    4.25 & 1.3330041486291346 \\ \hline
    4.75 & 1.2520373376623395 \\ \hline
\end{tabular}
\end{center}

Также программа была запущена на значениях, полученных при решении задачи Коши в предыдущей лабораторной работе. Было получено:

\[
  A =
  \begin{pmatrix}
    11.0 & 5.500000000000001 & 3.8500000000000005 & 3.0250000000000004\\
    5.500000000000001 & 3.8500000000000005 & 3.0250000000000004 & 2.5333\\
    3.8500000000000005 & 3.0250000000000004 & 2.5333 & 2.2082500000000005\\
    3.0250000000000004 & 2.5333 & 2.2082500000000005 & 1.9784050000000004
  \end{pmatrix}
\]
\[
  b =
  \begin{pmatrix}
    24.07015680910638 \\
    14.8400426642202 \\
    11.28159240138787 \\
    9.30124507302068
  \end{pmatrix}
\]
\[
  \lambda =
  \begin{pmatrix}
    0.999382178223316 \\
    2.0203891611643496 \\
    0.40411000986154677 \\
    0.1352029501748844
  \end{pmatrix}
\]

СКО $\Delta = 0.0005739961457411783$

Относительная ошибка $\delta = 7.419259034640004e-05$

Таблица значений аппроксимирующего многочлена z(x) в средних точках отрезков между узловыми точками

\begin{center}
\begin{tabular}{|c|c|}
    \hline
    $x_n$ & $z(x_n)$ \\ \hline
    0.05 & 1.101428811674959 \\ \hline
    0.15000000000000002 & 1.3119893375766936 \\ \hline
    0.25 & 1.5318488902272327 \\ \hline
    0.35000000000000003 & 1.761818687327626 \\ \hline
    0.45 & 2.002709946578923 \\ \hline
    0.55 & 2.2553338856821727 \\ \hline
    0.6500000000000001 & 2.520501722338425 \\ \hline
    0.75 & 2.7990246742487277 \\ \hline
    0.8500000000000001 & 3.0917139591141316 \\ \hline
    0.95 & 3.3993807946356855 \\ \hline
\end{tabular}
\end{center}

\textbf{Выводы:}

В результате выполнения лабораторной работы был изучен метод наименьших квадратов для решения задачи аппроксимации алгебраическими многочленами, была написала реализация данного метода на языке программирования Python. На тестовом примере среднеквадратичное отклонение составило $\Delta = 0.3179591701941208$, относительная ошибка составила $\delta = 0.09548667876045708$. Дальнейшее увеличение точности решения можно осуществить увеличением количества точек разбиения и повышением степени многочлена.

\end{document}