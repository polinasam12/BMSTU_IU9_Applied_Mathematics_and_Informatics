\documentclass [12pt]{article}


\usepackage{ucs}
\usepackage[utf8x]{inputenc} %Поддержка UTF8
\usepackage{cmap} % Улучшенный поиск русских слов в полученном pdf-файле
\usepackage[english,russian]{babel} %Пакет для поддержки русского и английского языка
\usepackage{graphicx} %Поддержка графиков
\usepackage{float} %Поддержка float-графиков
\usepackage[left=20mm,right=15mm, top=20mm,bottom=20mm,bindingoffset=0cm]{geometry}
\usepackage{mathtools}
\usepackage{setspace,amsmath}
\usepackage{amsmath,amssymb}
\usepackage{dsfont}
\DeclarePairedDelimiter{\abs}{\lvert}{\rvert}
\renewcommand{\baselinestretch}{1.2}

\usepackage{color}
\definecolor{deepblue}{rgb}{0,0,0.5}
\definecolor{deepred}{rgb}{0.6,0,0}
\definecolor{deepgreen}{rgb}{0,0.5,0}
\definecolor{gray}{rgb}{0.5,0.5,0.5}

\DeclareFixedFont{\ttb}{T1}{txtt}{bx}{n}{12} % for bold
\DeclareFixedFont{\ttm}{T1}{txtt}{m}{n}{12}  % for normal

\usepackage{listings}


\lstset{
	language=Python,
	basicstyle=\ttm,
	otherkeywords={self},             % Add keywords here
	keywordstyle=\ttb\color{deepblue},
	emph={MyClass,__init__},          % Custom highlighting
	emphstyle=\ttb\color{deepred},    % Custom highlighting style
	stringstyle=\color{deepgreen},
	frame=tb,                         % Any extra options here
	showstringspaces=false            %
}

\usepackage{hyperref}
\usepackage{wasysym}

\hypersetup{
    bookmarks=true,         % show bookmarks bar?
    unicode=false,          % non-Latin characters in Acrobat’s bookmarks
    pdftoolbar=true,        % show Acrobat’s toolbar?
    pdfmenubar=true,        % show Acrobat’s menu?
    pdffitwindow=false,     % window fit to page when opened
    pdfstartview={FitH},    % fits the width of the page to the window
    pdftitle={My title},    % title
    pdfauthor={Author},     % author
    pdfsubject={Subject},   % subject of the document
    pdfcreator={Creator},   % creator of the document
    pdfproducer={Producer}, % producer of the document
    pdfkeywords={keyword1} {key2} {key3}, % list of keywords
    pdfnewwindow=true,      % links in new PDF window
    colorlinks=true,       % false: boxed links; true: colored links
    linkcolor=black,          % color of internal links (change box color with linkbordercolor)
    citecolor=green,        % color of links to bibliography
    filecolor=magenta,      % color of file links
    urlcolor=cyan           % color of external links
}


\title{}
\date{}
\author{}

\begin{document}
\begin{titlepage}
\thispagestyle{empty}
\begin{center}
Федеральное государственное бюджетное образовательное учреждение высшего профессионального образования \\Московский государственный технический университет имени Н.Э. Баумана

\end{center}
\bigskip
\begin{flushleft}
Факультет: \underline{Информатика и системы управления}\\
Кафедра: \underline{Теоретическая информатика и компьютерные технологии}
\end{flushleft}
\vfill
\centerline{\large{Лабораторная работа №1}}
\centerline{\large{по курсу <<Численные методы>>}}
\centerline{\large{<<Решение СЛАУ с трехдиагональной матрицей методом прогонки>>}}
\vfill
\hfill\parbox{5cm} {
           Выполнила:\\
           студентка группы ИУ9-62Б \hfill \\
           Самохвалова П. С.\hfill \medskip\\
           Проверила:\\
           Домрачева А. Б.\hfill
       }
\centerline{Москва, 2023}
\clearpage
\end{titlepage}

\textsc{\textbf{Цель:}}

Изучение погрешности решения СЛАУ с трехдиагональной матрицей методом прогонки.

\textsc{\textbf{Постановка задачи:}}

\begin{enumerate}
    \item Известно точное решение $\overline{x}$, найти приближенное $\overline{x}^{*}$, определить $\overline{e_1} = \overline{x} - \overline{x}^{*}$
    \item Неизвестно точное решение $\overline{x}$, тогда найти приближенное решение $\overline{x}^{*}$ и далее $\overline{e_2} = A^{-1}(\overline{d} - \overline{d}^{*})$
    \item Сравнить $\overline{e_1}$ и $\overline{e_2}$, обосновать разницу
\end{enumerate}

\textbf{Описание алгоритма:}

$$ A=\left(\begin{array}{cccccc}
b_1 & c_1 & 0 & ... & ... &  0 \\
a_1 & b_2 & c_2 & ... & ... &  0 \\
0 & a_2 & b_3 & c_3 & ... &  0 \\
\vdots & ... & \ddots & \ddots & \ddots &  \vdots \\
0 & ... & ... & a_{n-2} & b_{n-1} & c_{n-1} \\
0 & ... & ... & ... & a_{n-1} & b_n \\
\end{array}\right) $$

$$\overline{d} = \left(\begin{array}{c}
d_1 \\
d_2 \\
\vdots \\
d_n \\
\end{array}\right)$$

\begin{equation*}
\begin{cases}
b_1x_1+c_1x_2=d_1\\
a_1x_1+b_2x_2+c_2x_3=d_2 \\
... \\
a_{n-1}x_{n-1}+b_nx_n=d_n
\end{cases}
\end{equation*}

$$ x_1= -\frac{c_1}{b_1}x_2 + \frac{d_1}{b_1} $$
$$ b_1 \neq 0 $$
$$ \alpha_1=-\frac{c_1}{b_1}, \beta_1=\frac{d_1}{b_1} $$
$$ a_1(\alpha_1x_2 + \beta_1) + b_2x_2 + c_2x_3 = (a_1\alpha_1 + b_2)x_2 + c_2x_3 = d_2 - a_1\beta_1 $$
$$ x_2= -\frac{c_2}{a_1\alpha_1 + b_2}x_3 + \frac{d_2 - a_1\beta_1}{a_1\alpha_1 + b_2 }$$
$$ x_i=\alpha_ix_{i + 1}+\beta_i $$

\begin{equation*}
\begin{cases}
$$ \alpha_1=-\frac{c_1}{b_1}, \beta_1=\frac{d_1}{b_1}  $$\\
$$ \alpha_i=-\frac{c_i}{a_{i - 1}\alpha_{i - 1} + b_i}, \beta_i=\frac{d_i - a_{i - 1}\beta_{i - 1}}{a_{i - 1}\alpha_{i - 1} + b_i}, i=\overline{2, n - 1} $$\\
$$x_i=\alpha_ix_{i + 1}+\beta_i, x_n = \frac{d_n - a_{n - 1}\beta_{n - 1}}{a_{n - 1}\alpha_{n - 1} + b_n} $$
\end{cases}
\end{equation*}

$$ x_{n - 1} = \alpha_{n - 1}x_n+\beta_{n - 1} $$
$$ a_{n - 1}(\alpha_{n - 1}x_n + \beta_{n - 1}) + b_nx_n = d_n $$
$$ (a_{n - 1}\alpha_{n - 1} + b_n)x_n = d_n - a_{n - 1}\beta_{n - 1} $$
$$ x_n = \frac{d_n - a_{n - 1}\beta_{n - 1}}{a_{n - 1}\alpha_{n - 1} + b_n} $$
$$ x_i=\alpha_ix_{i + 1}+\beta_i $$

$$\left|\frac{a_{i - 1}}{b_i}\right| \leq 1$$
$$\left|\frac{c_i}{b_i}\right| \leq 1$$

$$\left|b_i\right| \geq \left|a_{i - 1}\right| + \left|c_i\right|$$

$$ A\overline{x} = \overline{d} $$
$$ A\overline{x}^* = \overline{d}^* $$
$$ A(\overline{x}-\overline{x}^*) = \overline{d}-\overline{d}^* $$
$$ A\overline{e} = \overline{r} $$
$$ \overline{e} = A^{-1} \overline{r} $$
$$ \overline{x} = \overline{x}^* + \overline{e} $$

\textsc{\textbf{Практическая реализация:}}

Листинг 1. Решение СЛАУ с трехдиагональной матрицей методом прогонки
\begin{lstlisting}[language=python]

from decimal import *

getcontext().prec = 32


def det(m):
    l = len(m)
    if l == 2:
        return m[0][0] * m[1][1] - m[1][0] * m[0][1]
    else:
        d = 0
        for i in range(l):
            d += m[0][i] * alg_add(m, 0, i)
        return d


def alg_add(m, i, j):
    l = len(m)
    m1 = []
    for k in range(l):
        if k != i:
            m1.append([])
            for q in range(l):
                if q != j:
                    m1[-1].append(m[k][q])
    return (-1) ** (i + j) * det(m1)


def module_v(v):
    s = 0
    for i in range(len(v)):
        s += v[i] ** Decimal(2)
    return s ** Decimal(1/2)


def print_v(v):
    for i in range(len(v)):
        print(v[i])


n = 4

m = [[4, 1, 0, 0],
     [1, 4, 1, 0],
     [0, 1, 4, 1],
     [0, 0, 1, 4]]

x_accurate = [1, 1, 1, 1]

a = [0]
b = [0]
c = [0]

for i in range(n):
    for j in range(n):
        m[i][j] = Decimal(m[i][j])
        if i == j:
            b.append(m[i][j])
        elif i == j + 1:
            a.append(m[i][j])
        elif i == j - 1:
            c.append(m[i][j])

d = [0, 5, 6, 6, 5]
d = list(map(str, d))
d = list(map(Decimal, d))

alpha = [0] * n
beta = [0] * n

for i in range(1, n):
    alpha[i] = -c[i] / (a[i - 1] * alpha[i - 1] + b[i])
    beta[i] = (d[i] - a[i - 1] * beta[i - 1]) / \
              (a[i - 1] * alpha[i - 1] + b[i])

x_inaccurate = [0] * (n + 1)

x_inaccurate[n] = (d[n] - a[n - 1] * beta[n - 1]) / \
                  (a[n - 1] * alpha[n - 1] + b[n])

for i in range(n - 1, 0, -1):
    x_inaccurate[i] = alpha[i] * x_inaccurate[i + 1] + beta[i]

x_inaccurate = x_inaccurate[1:]

e1 = [0] * n

for i in range(n):
    e1[i] = x_accurate[i] - x_inaccurate[i]

print("x_inaccurate")
print_v(x_inaccurate)
print()
print("e1")
print_v(e1)
print()
print("|e1|")
print(module_v(e1))
print()

a = a[1:]
b = b[1:]
c = c[1:]
d = d[1:]

d_inaccurate = [0] * n

d_inaccurate[0] = b[0] * x_inaccurate[0] + c[0] * x_inaccurate[1]
if n > 1:
    d_inaccurate[n - 1] = a[n - 2] * x_inaccurate[-2] + b[n - 1] * \
                          x_inaccurate[-1]
for i in range(1, n - 1):
    d_inaccurate[i] = a[i - 1] * x_inaccurate[i - 1] + b[i] * \
                      x_inaccurate[i] + c[i] * x_inaccurate[i + 1]

r = [0] * n

for i in range(n):
    r[i] = d[i] - d_inaccurate[i]

m_inverse = []
for i in range(n):
    m_inverse.append([0] * n)

c = Decimal(1) / Decimal(det(m))

for i in range(n):
    for j in range(n):
        m_inverse[j][i] = c * alg_add(m, i, j)

e2 = [0] * n

for i in range(n):
    for j in range(n):
        e2[i] += m_inverse[i][j] * r[j]

x = [0] * n

for i in range(n):
    x[i] = x_inaccurate[i] + e2[i]

print("d_inaccurate")
print_v(d_inaccurate)
print()
print("x")
print_v(x)
print()
print("e2")
print_v(e2)
print()
print("|e2|")
print(module_v(e2))


\end{lstlisting}


\textsc{\textbf{Результаты работы:}}

$$A = \left(\begin{array}{cccc}
4 & 1 & 0 & 0 \\
1 & 4 & 1 & 0 \\
0 & 1 & 4 & 1 \\
0 & 0 & 1 & 4
\end{array}\right) $$

$$\overline{d} = \left(\begin{array}{c}
5 \\
6 \\
6 \\
5 \\
\end{array}\right)$$

Точное решение:

$$\overline{x_{accurate}} = \left(\begin{array}{c}
1 \\
1 \\
1 \\
1 \\
\end{array}\right)
$$

В результате работы программы получаем:

$$\overline{x}^{*} = \left(\begin{array}{c}
1.0000000000000000000000000000000 \\
1.0000000000000000000000000000000 \\
0.99999999999999999999999999999995 \\
1.0000000000000000000000000000000 \\
\end{array}\right)
$$

$$\overline{e_1} = \overline{x_{accurate}} - \overline{x}^{*} = \left(\begin{array}{c}
0E-31 \\
0E-31 \\
5E-32 \\
0E-31 \\
\end{array}\right)
$$

$$|\overline{e_1}| = 5.0000000000000000000000000000000E-32$$

$$\overline{d}^{*} = A\overline{x}^{*} = \left(\begin{array}{c}
5.0000000000000000000000000000000 \\
6.0000000000000000000000000000000 \\
5.9999999999999999999999999999998 \\
5.0000000000000000000000000000000 \\
\end{array}\right)
$$

$$\overline{e_2} = A^{-1}(\overline{d} - \overline{d}^{*}) = \left(\begin{array}{c}
3.8277511961722488038277511961722E-33 \\
-1.5311004784688995215311004784689E-32 \\
5.7416267942583732057416267942584E-32 \\
-1.4354066985645933014354066985646E-32 \\
\end{array}\right)
$$

$$|\overline{e_2}| = 6.1251494759063196724901316333084E-32$$

$$\overline{x} = \overline{x}^* + \overline{e_2} = \left(\begin{array}{c}
1.0000000000000000000000000000000 \\
0.99999999999999999999999999999998 \\
1.0000000000000000000000000000000 \\
0.99999999999999999999999999999999 \\
\end{array}\right)$$


\textbf{Выводы:}

В результате выполнения лабораторной работы было рассмотрено решение СЛАУ с трёхдиагональной матрицей методом прогонки, была написана реализация на языке программирования python. Для данного метода можно сделать вывод, что отсутствует методологическая погрешность, но присутствует вычислительная погрешность из-за использования чисел с плавающей запятой.
В вычислениях использовались числа, имеющие 32 знака после запятой. Были получны значения:
$$\overline{e_1} = \left(\begin{array}{c}
0E-31 \\
0E-31 \\
5E-32 \\
0E-31 \\
\end{array}\right)
$$
$$|\overline{e_1}| = 5.0000000000000000000000000000000E-32$$
$$\overline{e_2} = \left(\begin{array}{c}
3.8277511961722488038277511961722E-33 \\
-1.5311004784688995215311004784689E-32 \\
5.7416267942583732057416267942584E-32 \\
-1.4354066985645933014354066985646E-32 \\
\end{array}\right)
$$
$$|\overline{e_2}| = 6.1251494759063196724901316333084E-32$$
Значение вектора ошибки $\overline{e_2}$ можно объяснить накоплением вычислительной ошибки при выполнении арифметических операций над числами с плавающей запятой при вычислении обратной матрицы.


\end{document}