\documentclass [12pt]{article}


\usepackage{ucs}
\usepackage[utf8x]{inputenc} %Поддержка UTF8
\usepackage{cmap} % Улучшенный поиск русских слов в полученном pdf-файле
\usepackage[english,russian]{babel} %Пакет для поддержки русского и английского языка
\usepackage{graphicx} %Поддержка графиков
\usepackage{float} %Поддержка float-графиков
\usepackage[left=20mm,right=15mm, top=20mm,bottom=20mm,bindingoffset=0cm]{geometry}
\usepackage{mathtools}
\usepackage{setspace,amsmath}
\usepackage{amsmath,amssymb}
\usepackage{dsfont}
\usepackage{tabularx}
\DeclarePairedDelimiter{\abs}{\lvert}{\rvert}
\renewcommand{\baselinestretch}{1.2}

\usepackage{color}
\definecolor{deepblue}{rgb}{0,0,0.5}
\definecolor{deepred}{rgb}{0.6,0,0}
\definecolor{deepgreen}{rgb}{0,0.5,0}
\definecolor{gray}{rgb}{0.5,0.5,0.5}

\DeclareFixedFont{\ttb}{T1}{txtt}{bx}{n}{12} % for bold
\DeclareFixedFont{\ttm}{T1}{txtt}{m}{n}{12}  % for normal

\usepackage{listings}


\lstset{
	language=Python,
	basicstyle=\ttm,
	otherkeywords={self},             % Add keywords here
	keywordstyle=\ttb\color{deepblue},
	emph={MyClass,__init__},          % Custom highlighting
	emphstyle=\ttb\color{deepred},    % Custom highlighting style
	stringstyle=\color{deepgreen},
	frame=tb,                         % Any extra options here
	showstringspaces=false            %
}

\usepackage{hyperref}
\usepackage{wasysym}

\hypersetup{
    bookmarks=true,         % show bookmarks bar?
    unicode=false,          % non-Latin characters in Acrobat’s bookmarks
    pdftoolbar=true,        % show Acrobat’s toolbar?
    pdfmenubar=true,        % show Acrobat’s menu?
    pdffitwindow=false,     % window fit to page when opened
    pdfstartview={FitH},    % fits the width of the page to the window
    pdftitle={My title},    % title
    pdfauthor={Author},     % author
    pdfsubject={Subject},   % subject of the document
    pdfcreator={Creator},   % creator of the document
    pdfproducer={Producer}, % producer of the document
    pdfkeywords={keyword1} {key2} {key3}, % list of keywords
    pdfnewwindow=true,      % links in new PDF window
    colorlinks=true,       % false: boxed links; true: colored links
    linkcolor=black,          % color of internal links (change box color with linkbordercolor)
    citecolor=green,        % color of links to bibliography
    filecolor=magenta,      % color of file links
    urlcolor=cyan           % color of external links
}


\title{}
\date{}
\author{}

\begin{document}
\begin{titlepage}
\thispagestyle{empty}
\begin{center}
Федеральное государственное бюджетное образовательное учреждение высшего профессионального образования \\Московский государственный технический университет имени Н.Э. Баумана

\end{center}
\bigskip
\begin{flushleft}
Факультет: \underline{Информатика и системы управления}\\
Кафедра: \underline{Теоретическая информатика и компьютерные технологии}
\end{flushleft}
\vfill
\centerline{\large{Лабораторная работа №3}}
\centerline{\large{по курсу <<Численные методы>>}}
\centerline{\large{<<Интерполяция функции кубическим сплайном>>}}
\vfill
\hfill\parbox{5cm} {
           Выполнила:\\
           студентка группы ИУ9-62Б \hfill \\
           Самохвалова П. С.\hfill \medskip\\
           Проверила:\\
           Домрачева А. Б.\hfill
       }
\centerline{Москва, 2023}
\clearpage
\end{titlepage}

\textsc{\textbf{Цель:}}

Анализ метода интерполяции функции, основанного на построении кубического сплайна.

\textsc{\textbf{Постановка задачи:}}

\textbf{Дано:}

Вариант 21
$$y(x) = \frac{1}{x + x^2}$$
$$a = 0.25$$
$$b = 2$$
$$h = \frac{b - a}{32}$$

\begin{center}
 \begin{tabular}{|c|c|}
\hline
x & y\\
\hline
0.25 & 3.2\\
\hline
0.3046875 & 2.5155842161830186\\
\hline
0.359375 & 2.046976511744128\\
\hline
0.4140625 & 1.7079120191806525\\
\hline
0.46875 & 1.4524822695035462\\
\hline
0.5234375 & 1.2540375047837735\\
\hline
0.578125 & 1.096066363393096\\
\hline
0.6328125 & 0.9678067221926872\\
\hline
0.6875 & 0.8619528619528619\\
\hline
0.7421875 & 0.773377389662497\\
\hline
0.796875 & 0.6983802216538789\\
\hline
0.8515625 & 0.6342275384198506\\
\hline
0.90625 & 0.5788581119276427\\
\hline
0.9609375 & 0.5306902471415152\\
\hline
1.015625 & 0.4884913536076327\\
\hline
1.0703125 & 0.4512877014185374\\
\hline
1.125 & 0.41830065359477125\\
\hline
1.1796875 & 0.38890075719812955\\
\hline
1.234375 & 0.36257413472603345\\
\hline
1.2890625 & 0.3388975074981901\\
\hline
1.34375 & 0.31751937984496126\\
\hline
1.3984375 & 0.2981456881334959\\
\hline
1.453125 & 0.28052873090884184\\
\hline
1.5078125 & 0.2644585411521637\\
\hline
1.5625 & 0.2497560975609756\\
\hline
1.6171875 & 0.23626793568389934\\
\hline
\end{tabular}
\end{center}

\begin{center}
 \begin{tabular}{|c|c|}
\hline
x & y\\
\hline
1.671875 & 0.22386183527354211\\
\hline
1.7265625 & 0.21242334271156116\\
\hline
1.78125 & 0.20185294697417702\\
\hline
1.8359375 & 0.19206377117402262\\
\hline
1.890625 & 0.1829796738887648\\
\hline
1.9453125 & 0.17453367848050025\\
\hline
2.0 & 0.16666666666666666\\
\hline
\end{tabular}
\end{center}

\textbf{Найти:}

\begin{enumerate}
    \item Кубический сплайн $S^3(x)$.
    \item Значения сплайна в узлах интерполяции, абсолютную погрешность.
    \item Значения сплайна в точках между узлами интерполяции, абсолютную погрешность.
\end{enumerate}

\textbf{Описание алгоритма:}\\

$$ S_i(x) = a_i + b_i(x - x_{i-1}) + c_i(x - x_{i - 1}) ^ 2 + d_i(x - x_{i - 1})^3, x \in [x_{i - 1}, x_i]$$
$$h=\frac{x_n - x_0}{n}$$

\begin{equation*}
\begin{cases}
c_1 = 0\\
4c_2 + c_3 = \frac{3}{h^2}(y_2 - 2y_1 + y_0)\\
c_i + 4c_{i+1}+c_{i+2} = \frac{3}{h^2}(y_{i+1}-2y_i+y_{i-1}), i = 2, , n - 2\\
c_{n - 1} + 4c_n =\frac{3}{h^2}(y_n - 2y_{n - 1} + y_{n - 2})\\
c_{n + 1} = 0
\end{cases}
\end{equation*}

Эту систему можно решить методом прогонки. Остальные коэффициенты сплайна можно найти по следующим формулам:
$$a_i = y_{i - 1}$$
$$b_i=\frac{y_i - y_{i - 1}}{h} - \frac{h}{3}(c_{i + 1} + 2c_i)$$
$$d_i=\frac{c_{i + 1} - c_i}{3h}$$

Листинг 1. Интерполяция функции кубическим сплайном

\begin{lstlisting}[language=python]
def f(x):
    return 1 / (x + x ** 2)

n = 32
a = 0.25
b = 2
h = (b - a) / n

y = [0] * (n + 1)
x = [0] * (n + 1)

for i in range(n + 1):
    x[i] = a + h * i
    y[i] = f(x[i])

nm = n - 1

am = [0] + [1] * (nm - 1)
bm = [0] + [4] * nm
cm = [0] + [1] * (nm - 1)

dm = [0] * (nm + 1)

k = 3 / h ** 2

for i in range(1, nm + 1):
    dm[i] = k * (y[i + 1] - 2 * y[i] + y[i - 1])

alpha = [0] * nm
beta = [0] * nm

for i in range(1, nm):
    alpha[i] = -cm[i] / (am[i - 1] * alpha[i - 1] + bm[i])
    beta[i] = (dm[i] - am[i - 1] * beta[i - 1]) / (am[i - 1] *
                                                   alpha[i - 1] + bm[i])

xr = [0] * (nm + 1)

xr[nm] = (dm[nm] - am[nm - 1] * beta[nm - 1]) / (am[nm - 1] *
                                                 alpha[nm - 1] + bm[nm])

for i in range(nm - 1, 0, -1):
    xr[i] = alpha[i] * xr[i + 1] + beta[i]

ci = [0, 0] + xr[1:] + [0]

ai = [0] * (n + 1)
bi = [0] * (n + 1)
di = [0] * (n + 1)

for i in range(1, n + 1):
    ai[i] = y[i - 1]
    bi[i] = (y[i] - y[i - 1]) / h - h * (ci[i + 1] + 2 * ci[i]) / 3
    di[i] = (ci[i + 1] - ci[i]) / (3 * h)

for i in range(1, n + 1):
    print("a[" + str(i) + "] = " + str(ai[i]) + ", b[" + str(i) + "] = "
          + str(bi[i]) + ", c[" + str(i) + "] = " + str(ci[i]) +
          ", d[" + str(i) + "] =", di[i])

print()

for i in range(n + 1):
    if i == 0:
        sr = ai[1]
        print("x[0] = " + str(x[0]))
        print("y(" + str(x[0]) + ") = " + str(y[0]))
        print("S1(" + str(x[0]) + ") = " + str(sr))
        print("e = " + str(abs(y[0] - sr)))
        print()
    else:
        sr1 = ai[i] + bi[i] * (h / 2) + ci[i] * ((h / 2) ** 2) + \
              di[i] * ((h / 2) ** 3)
        print("x[" + str(i) + "] - " + str(h / 2) + " = " +
              str(x[i] - h / 2))
        print("y(" + str(x[i] - h / 2) + ") = " + str(f(x[i] -
                                                        h / 2)))
        print("S" + str(i) + "(" + str(x[i] - h / 2) + ") = " +
              str(sr1))
        print("e = " + str(abs(f(x[i] - h / 2) - sr1)))
        print()
        sr = ai[i] + bi[i] * h + ci[i] * (h ** 2) + di[i] * (h ** 3)
        print("x[" + str(i) + "] = " + str(x[i]))
        print("y(" + str(x[i]) + ") = " + str(y[i]))
        print("S" + str(i) + "(" + str(x[i]) + ") = " + str(sr))
        print("e = " + str(abs(y[i] - sr)))
        print()


\end{lstlisting}


\textsc{\textbf{Результаты работы:}}

Значения сплайна в узлах интерполяции и в точках между узлами интерполяции, абсолютная погрешность:

\begin{center}
\begin{tabular}{|c|c|c|c|}
\hline
x & y & $S^3(x)$ & $\varepsilon$ \\
\hline
0.25 & 3.2 & 3.2 & 0.0\\
\hline
0.27734375 & 2.8227591850798985 & 2.8390845668900253 & 0.01632538181012677\\
\hline
0.3046875 & 2.5155842161830186 & 2.5155842161830186 & 0.0\\
\hline
0.33203125 & 2.261031568052441 & 2.2564749578012413 & 0.0045566102511998\\
\hline
0.359375 & 2.046976511744128 & 2.046976511744128 & 0.0\\
\hline
0.38671875 & 1.8647318252952056 & 1.8658666970931368 & 0.0011348717979311473\\
\hline
0.4140625 & 1.7079120191806525 & 1.7079120191806525 & 0.0\\
\hline
0.44140625 & 1.57171978799434 & 1.571371090945776 & 0.0003486970485639951\\
\hline
0.46875 & 1.4524822695035462 & 1.452482269503546 & 2.220446049250313e-16\\
\hline
0.49609375 & 1.3473407207911021 & 1.3474092711568177 & 6.8550365715625e-05\\
\hline
0.5234375 & 1.2540375047837735 & 1.2540375047837735 & 0.0\\
\hline
0.55078125 & 1.1707665648391303 & 1.1707334733247152 & 3.3091514415106715e-05\\
\hline
0.578125 & 1.096066363393096 & 1.0960663633930963 & 2.220446049250313e-16\\
\hline
0.60546875 & 1.028741856997096 & 1.0287415805148503 & 2.764822457645977e-07\\
\hline
0.6328125 & 0.9678067221926872 & 0.9678067221926872 & 0.0\\
\hline
0.66015625 & 0.9124399582318135 & 0.9124341215170898 & 5.836714723650438e-06\\
\hline
0.6875 & 0.8619528619528619 & 0.861952861952862 & 1.1102230246251565e-16\\
\hline
0.71484375 & 0.8157635958524715 & 0.8157612069671932 & 2.3888852782594228e-06\\
\hline
0.7421875 & 0.773377389662497 & 0.773377389662497 & 0.0\\
\hline
0.76953125 & 0.73437097298327 & 0.7343688919891151 & 2.080994154884941e-06\\
\hline
0.796875 & 0.6983802216538789 & 0.6983802216538788 & 1.1102230246251565e-16\\
\hline
0.82421875 & 0.6650902706597521 & 0.6650889076575162 & 1.363002235965638e-06\\
\hline
0.8515625 & 0.6342275384198506 & 0.6342275384198505 & 1.1102230246251565e-16\\
\hline
0.87890625 & 0.6055532455532455 & 0.6055522252915599 & 1.020261685580337e-06\\
\hline
0.90625 & 0.5788581119276427 & 0.5788581119276427 & 0.0\\
\hline
0.93359375 & 0.5539579899412536 & 0.5539572445247115 & 7.454165420472947e-07\\
\hline
0.9609375 & 0.5306902471415152 & 0.5306902471415151 & 1.1102230246251565e-16\\
\hline
\end{tabular}
\end{center}

\begin{center}
\begin{tabular}{|c|c|c|c|}
\hline
x & y & $S^3(x)$ & $\varepsilon$ \\
\hline
0.98828125 & 0.5089107526965219 & 0.5089101904368587 & 5.622596631615906e-07\\
\hline
1.015625 & 0.4884913536076327 & 0.4884913536076327 & 0.0\\
\hline
1.04296875 & 0.46931775051739816 & 0.4693173225499263 & 4.2796747184992157e-07\\
\hline
1.0703125 & 0.4512877014185374 & 0.4512877014185374 & 0.0\\
\hline
1.09765625 & 0.4343094958812965 & 0.43430916515571094 & 3.30725585584446e-07\\
\hline
1.125 & 0.41830065359477125 & 0.41830065359477125 & 0.0\\
\hline
1.15234375 & 0.4031868098065151 & 0.4031865513412127 & 2.584653024384487e-07\\
\hline
1.1796875 & 0.38890075719812955 & 0.38890075719812955 & 0.0\\
\hline
1.20703125 & 0.3753816192685511 & 0.3753814150270995 & 2.042414516401081e-07\\
\hline
1.234375 & 0.36257413472603345 & 0.3625741347260335 & 5.551115123125783e-17\\
\hline
1.26171875 & 0.3504280359539507 & 0.350427872968978 & 1.6298497274025436e-07\\
\hline
1.2890625 & 0.3388975074981901 & 0.3388975074981901 & 0.0\\
\hline
1.31640625 & 0.32794071286672904 & 0.3279405816220175 & 1.3124471154313255e-07\\
\hline
1.34375 & 0.31751937984496126 & 0.31751937984496126 & 0.0\\
\hline
1.37109375 & 0.3075984360992598 & 0.3075983295119532 & 1.0658730659196536e-07\\
\hline
1.3984375 & 0.2981456881334959 & 0.2981456881334959 & 0.0\\
\hline
1.42578125 & 0.2891315377318951 & 0.28913145056911294 & 8.716278215858964e-08\\
\hline
1.453125 & 0.28052873090884184 & 0.28052873090884184 & 0.0\\
\hline
1.48046875 & 0.2723121351255895 & 0.27231206308295663 & 7.204263285931489e-08\\
\hline
1.5078125 & 0.2644585411521637 & 0.26445854115216366 & 5.551115123125783e-17\\
\hline
1.53515625 & 0.2569464864716514 & 0.2569464274862605 & 5.8985390904986446e-08\\
\hline
1.5625 & 0.2497560975609756 & 0.24975609756097558 & 2.7755575615628914e-17\\
\hline
1.58984375 & 0.2428689487513017 & 0.24286889659636968 & 5.215493201204957e-08\\
\hline
1.6171875 & 0.23626793568389934 & 0.23626793568389934 & 0.0\\
\hline
1.64453125 & 0.22993716164298972 & 0.2299371287604896 & 3.288250011168614e-08\\
\hline
1.671875 & 0.22386183527354211 & 0.22386183527354211 & 0.0\\
\hline
1.69921875 & 0.21802817838548164 & 0.2180281098983832 & 6.848709843687573e-08\\
\hline
1.7265625 & 0.21242334271156116 & 0.21242334271156116 & 0.0\\
\hline
1.75390625 & 0.20703533462856782 & 0.20703542868540906 & 9.405684123220404e-08\\
\hline
1.78125 & 0.20185294697417702 & 0.20185294697417702 & 0.0\\
\hline
1.80859375 & 0.1968656971976317 & 0.19686520901242352 & 4.881852081750626e-07\\
\hline
1.8359375 & 0.19206377117402262 & 0.19206377117402265 & 2.7755575615628914e-17\\
\hline
1.86328125 & 0.18743797209137372 & 0.1874396770160116 & 1.7049246378852967e-06\\
\hline
1.890625 & 0.1829796738887648 & 0.1829796738887648 & 0.0\\
\hline
1.91796875 & 0.17868077878383867 & 0.1786743156608785 & 6.463122960181922e-06\\
\hline
1.9453125 & 0.17453367848050025 & 0.17453367848050025 & 0.0\\
\hline
1.97265625 & 0.17053121869348564 & 0.1705552531321976 & 2.403443871196176e-05\\
\hline
2.0 & 0.16666666666666666 & 0.16666666666666663 & 2.7755575615628914e-17\\
\hline
\end{tabular}
\end{center}

\textbf{Выводы:}

В результате выполнения лабораторной работы был изучен метод интерполяции функции, основанный на построении кубического сплайна, была написала реализация данного метода на языке программирования python. Были вычислены значения сплайна в узлах интерполяции. Погрешность сплайна в узлах интерполяции составляет от 0 до 2.220446049250313e-16. Она не является методологической, а является вычислительной, обусловленной использованием чисел с плавающей запятой. Также были вычислены значения сплайна в точках между узлами интерполяции. Они являются приближенными значениями функции в этих точках. Погрешность сплайна в точках между узлами интерполяции составляет от 3.288250011168614e-08 до 0.01632538181012677.

\end{document}