\documentclass [12pt]{article}


\usepackage{ucs}
\usepackage[utf8x]{inputenc} %Поддержка UTF8
\usepackage{cmap} % Улучшенный поиск русских слов в полученном pdf-файле
\usepackage[english,russian]{babel} %Пакет для поддержки русского и английского языка
\usepackage{graphicx} %Поддержка графиков
\usepackage{float} %Поддержка float-графиков
\usepackage[left=20mm,right=15mm, top=20mm,bottom=20mm,bindingoffset=0cm]{geometry}
\usepackage{mathtools}
\usepackage{setspace,amsmath}
\usepackage{amsmath,amssymb}
\usepackage{dsfont}
\usepackage{tabularx}
\DeclarePairedDelimiter{\abs}{\lvert}{\rvert}
\renewcommand{\baselinestretch}{1.2}

\usepackage{color}
\definecolor{deepblue}{rgb}{0,0,0.5}
\definecolor{deepred}{rgb}{0.6,0,0}
\definecolor{deepgreen}{rgb}{0,0.5,0}
\definecolor{gray}{rgb}{0.5,0.5,0.5}

\DeclareFixedFont{\ttb}{T1}{txtt}{bx}{n}{12} % for bold
\DeclareFixedFont{\ttm}{T1}{txtt}{m}{n}{12}  % for normal

\usepackage{listings}


\lstset{
	language=Python,
	basicstyle=\ttm,
	otherkeywords={self},             % Add keywords here
	keywordstyle=\ttb\color{deepblue},
	emph={MyClass,__init__},          % Custom highlighting
	emphstyle=\ttb\color{deepred},    % Custom highlighting style
	stringstyle=\color{deepgreen},
	frame=tb,                         % Any extra options here
	showstringspaces=false            %
}

\usepackage{hyperref}
\usepackage{wasysym}

\hypersetup{
    bookmarks=true,         % show bookmarks bar?
    unicode=false,          % non-Latin characters in Acrobat’s bookmarks
    pdftoolbar=true,        % show Acrobat’s toolbar?
    pdfmenubar=true,        % show Acrobat’s menu?
    pdffitwindow=false,     % window fit to page when opened
    pdfstartview={FitH},    % fits the width of the page to the window
    pdftitle={My title},    % title
    pdfauthor={Author},     % author
    pdfsubject={Subject},   % subject of the document
    pdfcreator={Creator},   % creator of the document
    pdfproducer={Producer}, % producer of the document
    pdfkeywords={keyword1} {key2} {key3}, % list of keywords
    pdfnewwindow=true,      % links in new PDF window
    colorlinks=true,       % false: boxed links; true: colored links
    linkcolor=black,          % color of internal links (change box color with linkbordercolor)
    citecolor=green,        % color of links to bibliography
    filecolor=magenta,      % color of file links
    urlcolor=cyan           % color of external links
}


\title{}
\date{}
\author{}

\begin{document}
\begin{titlepage}
\thispagestyle{empty}
\begin{center}
Федеральное государственное бюджетное образовательное учреждение высшего профессионального образования \\Московский государственный технический университет имени Н.Э. Баумана

\end{center}
\bigskip
\begin{flushleft}
Факультет: \underline{Информатика и системы управления}\\
Кафедра: \underline{Теоретическая информатика и компьютерные технологии}
\end{flushleft}
\vfill
\centerline{\large{Лабораторная работа №4}}
\centerline{\large{по курсу <<Численные методы>>}}
\centerline{\large{<<Сравнение численных решений краевой задачи методом прогонки и методом стрельбы>>}}
\vfill
\hfill\parbox{5cm} {
           Выполнила:\\
           студентка группы ИУ9-62Б \hfill \\
           Самохвалова П. С.\hfill \medskip\\
           Проверила:\\
           Домрачева А. Б.\hfill
       }
\centerline{Москва, 2023}
\clearpage
\end{titlepage}

\textsc{\textbf{Цель:}}

Сравнить погрешности, получаемые при численном решении краевой задачи для линейного дифференциального уравнения второго порядка методом прогонки и методом стрельбы.

\textsc{\textbf{Постановка задачи:}}

Вариант 21

$$p = 1, \quad q = -2, \quad f(x) = cos(x) - 3sin(x), \quad y_0 = 1, \quad y_0' = 2$$

Получить численное решение краевой задачи для линейного дифференциального уравнения второго порядка методом прогонки

\begin{enumerate}
    \item Написать и отладить процедуру для решения трехдиагональной линейной системы методом прогонки.
    \item Решить аналитически задачу Коши $$y'' + py' + qy = f(x), \quad y(0) = y_0, \quad y'(0) = y_0'$$ и по найденному решению задачи Коши $y(x)$ вычислить $b=y(1)$.
    \item Найти численное решение $(x_i, y_i), \quad i = 0, ..., n$ краевой задачи для того же уравнения с краевыми условиями y(0) = a, y(1) = b при n = 10.
    \item Найти погрешность численного решения $||y - \widetilde{y}|| = max \quad |y(x_i) - \widetilde{y}_i|$, \quad $0 \leq i \leq n$.
\end{enumerate}

Получить численное решение краевой задачи для линейного дифференциального уравнения второго порядка методом стрельбы

\begin{enumerate}
    \item Решить аналитически задачу Коши $$y'' + py' + qy = f(x), \quad y(0) = y_0, \quad y'(0) = y_0'$$ и по найденному решению задачи Коши $y(x)$ вычислить $b=y(1)$.
    \item Найти численное решение $(x_i, y_i), \quad i = 0, ..., n$ краевой задачи для того же уравнения с краевыми условиями y(0) = a, y(1) = b при n = 10.
    \item Найти погрешность численного решения $||y - \widetilde{y}|| = max \quad |y(x_i) - \widetilde{y}_i|$, \quad $0 \leq i \leq n$.
\end{enumerate}

Сравнить данные методы\\

\textbf{Описание методов:}\\

Численное решение краевой задачи для линейного дифференциального уравнения второго порядка методом прогонки\\
Краевая задача для линейного дифференциального уравнения второго порядка имеет вид

$$y'' + p(x)y' + q(x)y = f(x)$$
$$y(0) = a, \quad y(1) = b$$

Требуется найти частное решение уравнения, отвечающее краевым условиям.\\
Приближенным численным решением задачи называется сеточная функция $(x_i, y_i)$, i = 0, ..., n, заданная в (n + 1)-й точке $x_i = ih$, h = 1/n\\
Обозначим через $p_i = p(x_i), q_i = q(x_i), f_i = f(x_i)$ значения коэффициентов уравнения в точках $x_i$ (узлах сетки), i = 0, ..., n. Применяя разностную аппроксимацию произвоных по формулам численного дифференцирования, получим приближённую систему уравнений относительно ординат сеточной функции $y_i$:
$$\frac{y_{i + 1} - 2y{i} + 2y{i - 1}}{h^{2}} + p_i\frac{y_{i + 1} - y_{i - 1}}{2h} + q_iy_i = f_i$$
или после преобразований
$$y_{i - 1}(1 - \frac{h}{2}p_i) + y_i(h^2q_i - 2) + y_{i + 1}(1 + \frac{h}{2}p_i) = h^2f_i, \quad i = 1, ..., n - 1$$
с краевыми условиями
$$y_0 = a, \quad y_n = b$$
Система является разностной системой с краевыми условиями и представляет собой трехдиагональную систему линейных алгебраических уравнений (n + 1)-го порядка. Трехдиагональную линейную систему следует решать методом прогонки.\\

Решение краевой задачи методом стрельбы\\
Рассмотрим краевую задачу для линейного дифференциального уравнения второго порядка:
$$y'' + p(x)y' + q(x)y = f(x)$$
$$y(a) = A, \quad y(b) = B$$

Разобьём отрезок [a, b] на n частей точками $x_0$, $x_1$, ..., $x_n$, где $x_i = a + ih$, h = (b - a)/n, а производные в уравнении во всех внутренних точках заменим их разностными аналогами

$$y_i' = \frac{y_{i + 1} - y_{i - 1}}{2h}, y_i'' = \frac{y_{i + 1} - 2y_i + y_{i + 1}}{h^2}, \quad i = 1, 2, ..., n - 1$$

Будем искать решения, удовлетворяющие условиям

$$y_0[0] = A, \quad y_0[1] = D_0$$
$$y_1[0] = 0, \quad y_1[1] = D_1$$

(используются обозначения $y_0[i] = y_0(x_i), y_1[i] = y_1(x_i)$). Для уменьшения вычислительной погрешности обычно берут $D_0 = A + O(h), D_1 = O(h)$\\

Для определения $y_0$ и $y_1$ получим уравнения

$$\frac{y_0[i + 1] - 2y_0[i] + y_0[i - 1]}{h^2} + p_i\frac{y_0[i + 1] - y_0[i - 1]}{2h} + q_iy_0[i] = f_i$$
$$\frac{y_1[i + 1] - 2y_1[i] + y_1[i - 1]}{h^2} + p_i\frac{y_1[i + 1] - y_1[i - 1]}{2h} + q_iy_1[i] = 0$$

Отсюда имеем:

$$y_0[i + 1] = \frac{f_ih^2 + (2 - q_ih^2)y_0[i] - (1 - p_i\frac{h}{2})y_0[i - 1]}{1 + p_i\frac{h}{2}}$$
$$y_1[i + 1] = \frac{(2 - q_ih^2)y_1[i] - (1 - p_i\frac{h}{2})y_1[i - 1]}{1 + p_i\frac{h}{2}}$$
$$i = 1, 2, ..., n - 1$$
$$y_0[0] = A, \quad y_0[1] = D_0$$
$$y_1[0] = 0, \quad y_1[1] = D_1$$

После того как величины $y_0[2]$, ..., $y_0[n]$, $y_1[2]$, ..., $y_1[n]$ последовательно определены, находим $C_1$ из уравнения $y_0[n] + C_1y_1[n] = B$, т. е. $C_1 = \frac{B - y_0[n]}{y_1[n]}$. Искомое решение задачи находим теперь по формулам
$$y[i] = y_0[i] + C_1y_1[i], \quad i = 0, 1, ..., n$$

Листинг 1. Решение краевой задачи методом прогонки и методом стрельбы

\begin{lstlisting}[language=python]
import math


def func(x):
    return math.cos(x) - 3 * math.sin(x)


def y_solution(x):
    return math.exp(x) + math.sin(x)


n = 10

a = 0
b = 1
h = (b - a) / n

y_0 = 1
y_n = y_solution(1)

f = [0] * (n + 1)
x = [0] * (n + 1)

p = [1] * (n + 1)
q = [-2] * (n + 1)

for i in range(n + 1):
    x[i] = a + h * i
    f[i] = func(x[i])

a = [0]
b = [0]
c = [0]
d = [0]

for i in range(1, n):
    b.append(h * h * q[i] - 2)

for i in range(2, n):
    a.append(1 - h / 2 * p[i])

for i in range(1, n - 1):
    c.append(1 + h / 2 * p[i])

for i in range(1, n):
    if i == 1:
        d.append(h * h * f[i] - y_0 * (1 - h / 2 * p[i]))
    elif i == (n - 1):
        d.append(h * h * f[i] - y_n * (1 + h / 2 * p[i]))
    else:
        d.append(h * h * f[i])

nm = n - 1

alpha = [0] * nm
beta = [0] * nm

for i in range(1, nm):
    alpha[i] = -c[i] / (a[i - 1] * alpha[i - 1] + b[i])
    beta[i] = (d[i] - a[i - 1] * beta[i - 1]) / \
              (a[i - 1] * alpha[i - 1] + b[i])

y = [0] * (n + 1)

y[0] = y_0
y[n] = y_n

y[nm] = (d[nm] - a[nm - 1] * beta[nm - 1]) / \
                  (a[nm - 1] * alpha[nm - 1] + b[nm])

for i in range(nm - 1, 0, -1):
    y[i] = alpha[i] * y[i + 1] + beta[i]

y_right = [0] * (n + 1)

for i in range(n + 1):
    y_right[i] = y_solution(x[i])

for i in range(n + 1):
    print("{:20} {:20} {:20} {:20}".format(str(x[i]), str(y_right[i]),
                                           str(y[i]),
                                           str(abs(y_right[i] - y[i]))))

print()

e1 = 0

for i in range(n + 1):
    m = abs(y_right[i] - y[i])
    if m > e1:
        e1 = m

print(e1)

print()

y0 = [0] * (n + 1)
y1 = [0] * (n + 1)

y0[0] = y_0
y0[1] = y_0 + h
y1[1] = h

for i in range(1, n):
    y0[i + 1] = (f[i] * h * h + (2 - q[i] * h * h) * y0[i] -
                 (1 - p[i] * h / 2) * y0[i - 1]) / (1 + p[i] * h / 2)
    y1[i + 1] = ((2 - q[i] * h * h) * y1[i] - (1 - p[i] * h / 2) *
                 y1[i - 1]) / (1 + p[i] * h / 2)

c1 = (y_n - y0[n]) / y1[n]

y = [0] * (n + 1)

for i in range(n + 1):
    y[i] = y0[i] + c1 * y1[i]

for i in range(n + 1):
    print("{:20} {:20} {:20} {:20}".format(str(x[i]), str(y_right[i]),
                                           str(y[i]),
                                           str(abs(y_right[i] - y[i]))))

print()

e2 = 0

for i in range(n + 1):
    m = abs(y_right[i] - y[i])
    if m > e2:
        e2 = m

print(e2)

print()

print(e1 - e2)

\end{lstlisting}


\textsc{\textbf{Результаты работы:}}

В результате решения задачи Коши аналитически

$$y'' + py' + qy = f(x), \quad y(0) = y_0, \quad y'(0) = y_0'$$

при

$$p = 1, \quad q = -2, \quad f(x) = cos(x) - 3sin(x), \quad y_0 = 1, \quad y_0' = 2$$

было получено решение

$$y = e^x + sin(x)$$

Таблица значений, полученных при решении краевой задачи методом прогонки:
\begin{center}
\begin{tabular}{|c|c|c|c|}
\hline
Точка & Правильное значение & Полученное значение & Погрешность \\
\hline
0.0 & 1.0 & 1 & 0.0\\
\hline
0.1 & 1.205004334722476 & 1.2051111084120076 & 0.00010677368953171396\\
\hline
0.2 & 1.420072088955231 & 1.4202661696625054 & 0.0001940807072744466\\
\hline
0.30000000000000004 & 1.6453790142373428 & 1.6456406624584896 & 0.0002616482211468263\\
\hline
0.4 & 1.881243039949921 & 1.8815514625506524 & 0.00030842260073149497\\
\hline
0.5 & 2.128146809304331 & 2.1284794139881162 & 0.0003326046837850427\\
\hline
0.6000000000000001 & 2.3867612737855444 & 2.3870929393272875 & 0.00033166554174313134\\
\hline
0.7000000000000001 & 2.6579703947081676 & 2.65827273914253 & 0.00030234443436238934\\
\hline
0.8 & 2.9428970193919906 & 2.9431376494882873 & 0.00024063009629671228\\
\hline
0.9 & 3.242930020784433 & 3.243071746807069 & 0.00014172602263595735\\
\hline
1.0 & 3.5597528132669414 & 3.5597528132669414 & 0.0\\
\hline
\end{tabular}
\end{center}

Максимальная погрешность при решении краевой задачи методом прогонки равна $\varepsilon_1 = 0.0003326046837850427$

Таблица значений, полученных при решении краевой задачи методом стрельбы:
\begin{center}
\begin{tabular}{|c|c|c|c|}
\hline
Точка & Правильное значение & Полученное значение & Погрешность \\
\hline
0.0 & 1.0 & 1.0 & 0.0\\
\hline
0.1 & 1.205004334722476 & 1.2051111084120065 & 0.00010677368953060373\\
\hline
0.2 & 1.420072088955231 & 1.4202661696625036 & 0.00019408070727267024\\
\hline
0.30000000000000004 & 1.6453790142373428 & 1.6456406624584874 & 0.00026164822114460584\\
\hline
0.4 & 1.881243039949921 & 1.8815514625506493 & 0.00030842260072838634\\
\hline
0.5 & 2.128146809304331 & 2.1284794139881127 & 0.00033260468378149\\
\hline
0.6000000000000001 & 2.3867612737855444 & 2.3870929393272844 & 0.0003316655417400227\\
\hline
0.7000000000000001 & 2.6579703947081676 & 2.658272739142527 & 0.0003023444343592807\\
\hline
0.8 & 2.9428970193919906 & 2.9431376494882846 & 0.00024063009629404775\\
\hline
0.9 & 3.242930020784433 & 3.243071746807068 & 0.00014172602263462508\\
\hline
1.0 & 3.5597528132669414 & 3.5597528132669414 & 0.0\\
\hline
\end{tabular}
\end{center}

Максимальная погрешность при решении краевой задачи методом стрельбы равна $\varepsilon_2 = 0.00033260468378149$\\

\textbf{Выводы:}

В результате выполнения лабораторной работы были изучены методы прогонки и стрельбы решения краевой задачи для линейного дифференциального уравнения второго порядка, была написала реализация данных методов на языке программирования python. Максимальная погрешность при решении краевой задачи методом прогонки равна $\varepsilon_1 = 0.0003326046837850427$. Максимальная погрешность при решении краевой задачи методом стрельбы равна $\varepsilon_2 =\\
0.00033260468378149$. Максимальная погрешность решения краевой задачи методом прогонки на 3.552713678800501e-15 превосходит максимальную погрешность решения краевой задачи методом стрельбы: $\varepsilon_1 - \varepsilon_2 =$ 3.552713678800501e-15.

\end{document}