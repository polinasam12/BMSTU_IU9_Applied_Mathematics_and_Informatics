\documentclass [12pt]{article}


\usepackage{ucs}
\usepackage[utf8x]{inputenc} %Поддержка UTF8
\usepackage{cmap} % Улучшенный поиск русских слов в полученном pdf-файле
\usepackage[english,russian]{babel} %Пакет для поддержки русского и английского языка
\usepackage{graphicx} %Поддержка графиков
\usepackage{float} %Поддержка float-графиков
\usepackage[left=20mm,right=15mm, top=20mm,bottom=20mm,bindingoffset=0cm]{geometry}
\usepackage{mathtools}
\usepackage{setspace,amsmath}
\usepackage{amsmath,amssymb}
\usepackage{dsfont}
\usepackage{tabularx}
\DeclarePairedDelimiter{\abs}{\lvert}{\rvert}
\renewcommand{\baselinestretch}{1.2}

\usepackage{color}
\definecolor{deepblue}{rgb}{0,0,0.5}
\definecolor{deepred}{rgb}{0.6,0,0}
\definecolor{deepgreen}{rgb}{0,0.5,0}
\definecolor{gray}{rgb}{0.5,0.5,0.5}

\DeclareFixedFont{\ttb}{T1}{txtt}{bx}{n}{12} % for bold
\DeclareFixedFont{\ttm}{T1}{txtt}{m}{n}{12}  % for normal

\usepackage{listings}


\lstset{
	language=Python,
	basicstyle=\ttm,
	otherkeywords={self},             % Add keywords here
	keywordstyle=\ttb\color{deepblue},
	emph={MyClass,__init__},          % Custom highlighting
	emphstyle=\ttb\color{deepred},    % Custom highlighting style
	stringstyle=\color{deepgreen},
	frame=tb,                         % Any extra options here
	showstringspaces=false            %
}

\usepackage{hyperref}
\usepackage{wasysym}

\hypersetup{
    bookmarks=true,         % show bookmarks bar?
    unicode=false,          % non-Latin characters in Acrobat’s bookmarks
    pdftoolbar=true,        % show Acrobat’s toolbar?
    pdfmenubar=true,        % show Acrobat’s menu?
    pdffitwindow=false,     % window fit to page when opened
    pdfstartview={FitH},    % fits the width of the page to the window
    pdftitle={My title},    % title
    pdfauthor={Author},     % author
    pdfsubject={Subject},   % subject of the document
    pdfcreator={Creator},   % creator of the document
    pdfproducer={Producer}, % producer of the document
    pdfkeywords={keyword1} {key2} {key3}, % list of keywords
    pdfnewwindow=true,      % links in new PDF window
    colorlinks=true,       % false: boxed links; true: colored links
    linkcolor=black,          % color of internal links (change box color with linkbordercolor)
    citecolor=green,        % color of links to bibliography
    filecolor=magenta,      % color of file links
    urlcolor=cyan           % color of external links
}


\title{}
\date{}
\author{}

\begin{document}
\begin{titlepage}
\thispagestyle{empty}
\begin{center}
Федеральное государственное бюджетное образовательное учреждение высшего профессионального образования \\Московский государственный технический университет имени Н.Э. Баумана

\end{center}
\bigskip
\begin{flushleft}
Факультет: \underline{Информатика и системы управления}\\
Кафедра: \underline{Теоретическая информатика и компьютерные технологии}
\end{flushleft}
\vfill
\centerline{\large{Лабораторная работа №7}}
\centerline{\large{по курсу <<Численные методы>>}}
\centerline{\large{<<Решение системы линейных алгебраических уравнений}}
\centerline{\large{методом простой итерации и методом Зейделя>>}}
\vfill
\hfill\parbox{5cm} {
           Выполнила:\\
           студентка группы ИУ9-62Б \hfill \\
           Самохвалова П. С.\hfill \medskip\\
           Проверила:\\
           Домрачева А. Б.\hfill
       }
\centerline{Москва, 2023}
\clearpage
\end{titlepage}

\textsc{\textbf{Цель:}}

Анализ метода простой итерации и метода Зейделя для решения системы линейных алгебраических уравнений.

\textsc{\textbf{Постановка задачи:}}

\textbf{Дано:}  $Ax = b$, $x = (x_{i})$, $b = (b_{i})$, $A = (a_{ij})$, $i,j=\overline{1,n}$.

\textbf{Найти:} $x = (x_{i})$.

\textbf{Тестовый пример:}

Вариант 21

\[
    A =
    \begin{pmatrix}
      10.0 + \alpha & -1.0 & 0.2 & 2.0\\
      1.0 & 12.0 - \alpha & -2.0 & 0.1\\
      0.3 & -4.0 & 12.0 - \alpha & 1.0\\
      0.2 & -0.3 & -0.5 & 8.0 - \alpha
    \end{pmatrix}
\]
\[
  b =
  \begin{pmatrix}
    1.0 + \beta \\
    2.0 - \beta \\
    3.0 \\
    1.0
  \end{pmatrix}
\]
$$\alpha = 0.1 * 21, \quad \beta = 0.1 * 21$$

\textbf{Задание:}

\begin{enumerate}
    \item Решить систему методом простой итерации с относительной погрешностью 0.01:\\
    а) преобразовать систему к виду x = Fx + c, распечатать матрицу F и столбец c;\\
    б) найти норму ||F||;\\
    в) Для каждой итерации распечатать абсолютную и относительную ошибки.
    \item Решить систему методом Зейделя, обеспечить $||x^k - x^{k - 1}|| \leq 10^{-4}$
\end{enumerate}

\textbf{Описание методов:}\\

Преобразуем исходную СЛАУ к виду $x = Fx + c$, $x = (x_{i})$, $c = (c_{i})$, $F = (f_{ij})$, $i,j=\overline{1,n}$.

Зададимся некоторым начальным приближением к решению $x^0 = (x_{1}^0,...,x_{n}^0)^T$. Используя метод простой итерации каждое последующее приближение (итерацию) находим по предыдущему: $$ x^k = Fx^{k-1} + c, \quad k = 1,2,....$$

Последовательность приближений сходится к точному решению $$ \lim_{k\to \inf} x^k = x,$$ если какая-либо норма матрицы F меньше единицы, например, $$ ||F|| = \max_{1 \leq i \leq n} \sum_{j = 1}^{n} |f_{ij}| < 1.$$
При этом абсолютная погрешность очередной итерации $$ \Delta_{k} \leq \frac{||F||}{1 - ||F||}||x^k - x^{k-1}||,$$ а относительная погрешность - $$ \delta_{k} = \frac{\Delta_{k}}{||x^k||},$$ где $ ||x^k|| = \max_{1 \leq i \leq n} |x_{i}^k|$.

Всякую невырожденную систему Ax = b можно привести к виду x = Fx + c, для которого ||F|| < 1 и метод простой итерации сходится.

Если для всех строк $|a_{ii}| > \sum\limits_{j \neq i}{|a_{ij}|}, \quad i = \overline{1,n}$ (диагональные элементы превалируют) , то для приведения к требуемому виду достаточно разрешить каждое уравнение системы относительно ведущей неизвестной. Тогда
\begin{equation*}
    f_{ij} =
    \begin{cases}
       0 &\text{, i = j}\\
       -\frac{a_{ij}}{a_{ii}} &\text{, $i \neq j$}
    \end{cases}
\end{equation*}
и $ c_{i} = \frac{b_{i}}{a_{ii}}, \quad i,j = \overline{1,n}; \quad ||F|| < 1 $.

Метод Зейделя является модификацией метода простой итерации и сходится, как правило, быстрее. Для того, чтобы решить СЛАУ методом Зейделя неободимо представить матрицу F как сумму нижнетреугольной матрицы $F_{d}$ и верхнетреугольной матрицы $F_{u}$.

Тогда k-ая итерация метода будет удовлетворять рекуррентному соотношению $$ x^k = F_{d}x^k + F_{u}x^{k-1} + c, \quad k = 1,2,....$$

Листинг 1. Решение СЛАУ методом простой итерации и методом Зейлеля

\begin{lstlisting}[language=python]
def norm_v(x):
    norm = 0
    n = len(x)
    for i in range(n):
        if abs(x[i]) > norm:
            norm = abs(x[i])
    return norm


def sub_v(x, y):
    n = len(x)
    sub = [0] * n
    for i in range(n):
        sub[i] = x[i] - y[i]
    return sub


def sum_v(x, y):
    n = len(x)
    sum = [0] * n
    for i in range(n):
        sum[i] = x[i] + y[i]
    return sum


def mul_mv(a, x):
    n = len(x)
    mul = [0] * n
    for i in range(n):
        for j in range(n):
            mul[i] += a[i][j] * x[j]
    return mul


k = 21
a = 0.1 * k
b = 0.1 * k
am = [[10.0 + a, -1.0, 0.2, 2.0],
      [1.0, 12.0 - a, -2.0, 0.1],
      [0.3, -4.0, 12.0 - a, 1.0],
      [0.2, -0.3, -0.5, 8.0 - a]]
bm = [1.0 + b, 2.0 - b, 3.0, 1.0]

n = 4

fm = [[0] * n for i in range(n)]
cm = [0] * n

for i in range(n):
    for j in range(n):
        if i != j:
            fm[i][j] = -am[i][j] / am[i][i]
for i in range(n):
    cm[i] = bm[i] / am[i][i]

print("F =")
for i in range(n):
    for j in range(n):
        print("{:24} ".format(str(fm[i][j])), end = "")
    print()
print()

print("c =")
for i in range(n):
    print("{:24}".format(str(cm[i])))
print()

fm_norm = 0
for i in range(n):
    s = 0
    for j in range(n):
        s += abs(fm[i][j])
    if s > fm_norm:
        fm_norm = s

print("||F|| =", fm_norm)
print()

k = 0
xk = cm[:]
otn_delta = 1

print("A simple iteration method with a relative error of 0.01")
print()

while otn_delta > 0.01:
    k += 1
    xk_last = xk[:]
    xk = sum_v(mul_mv(fm, xk_last), cm)
    abs_delta = norm_v(sub_v(xk, xk_last))
    otn_delta = abs_delta / norm_v(xk)
    print("{:4} {:24} {:24}".format(str(k), str(abs_delta), str(otn_delta)))

print()

fd = [[0] * n for i in range(n)]
fu = [[0] * n for i in range(n)]

for i in range(n):
    for j in range(n):
        if j < i:
            fd[i][j] = fm[i][j]

for i in range(n):
    for j in range(n):
        if i <= j:
            fu[i][j] = fm[i][j]

print("Seidel's method with an absolute error of 0.0001")
print()

k = 0
xk = cm[:]
abs_delta = 1

while abs_delta > 0.0001:
    k += 1
    xk_last = xk[:]
    xk = mul_mv(fu, xk_last)
    xk = mul_mv(fd, xk)
    xk = sum_v(xk, cm)
    abs_delta = norm_v(sub_v(xk, xk_last))
    otn_delta = abs_delta / norm_v(xk)
    print("{:4} {:24} {:24}".format(str(k), str(abs_delta), str(otn_delta)))

\end{lstlisting}


\textsc{\textbf{Результаты работы:}}

\[
    F =
    \begin{pmatrix}
      0 & 0.08264462809917356 & -0.01652892561983471 & -0.1652892561983471\\
      -0.10101010101010101 & 0 & 0.20202020202020202 & -0.010101010101010102\\
      -0.0303030303030303 & 0.40404040404040403 & 0 & -0.10101010101010101\\
      -0.03389830508474576 & 0.05084745762711864 & 0.0847457627118644 & 0
    \end{pmatrix},
\]
\[
  c =
  \begin{pmatrix}
    0.25619834710743805 \\
    -0.010101010101010109 \\
    0.30303030303030304 \\
    0.1694915254237288
  \end{pmatrix},
\]
$||F|| = 0.5353535353535354$.

Таблица погрешностей, получаемых на каждой итерации при решении СЛАУ методом простой итерации с относительной погрешностью 0.01:
\begin{center}
\begin{tabular}{|c|c|c|}
\hline
Номер итерации & Абсолютная погрешность & Относительная погрешность \\
\hline
1 & 0.03385868773177919 & 0.12354248038332181\\
\hline
2 & 0.01294805238429747 & 0.04511309071213607\\
\hline
3 & 0.0025578023329712873 & 0.008946285079489141\\
\hline
\end{tabular}
\end{center}

Таблица погрешностей, получаемых на каждой итерации при решении СЛАУ методом Зейделя с абсолютной погрешностью 0.0001:
\begin{center}
\begin{tabular}{|c|c|c|}
\hline
Номер итерации & Абсолютная погрешность & Относительная погрешность \\
\hline
1 & 0.02506893293170409 & 0.07640655687051644\\
\hline
2 & 0.0020527513408289955 & 0.006217594985869593\\
\hline
3 & 0.00016866968910478342 & 0.0005106241027756813\\
\hline
4 & 1.3853778181194265e-05 & 4.193863411030351e-05\\
\hline
\end{tabular}
\end{center}


\textbf{Выводы:}

В результате выполнения лабораторной работы был изучен метод простой итерации и метод Зейделя решения системы линейных алгебраических уравнений, была написана реализация на языке программирования Python. При применении метода простой итерации к тестовому примеру потребовались 3 итерации для достижения относительной погрешности 0.01. На 3 итерации абсолютная погрешность составила 0.0025578023329712873, относительная погрешность составила 0.008946285079489141. При применении метода Зейделя к тестовому примеру потребовались 4 итерации для достижения абсолютной погрешности 0.0001. На 4 итерации абсолютная погрешность составила 1.3853778181194265e-05, относительная погрешность составила 4.193863411030351e-05.
\end{document}